\documentclass[12pt]{article}
\bibliographystyle{IEEEtran}
\usepackage{parskip}
\setlength{\parskip}{.5cm plus4mm minus3mm}
\usepackage[margin=1in]{geometry}
\usepackage[colorlinks]{hyperref}
\usepackage{color}
\definecolor{darkblue}{rgb}{0,0,0.5}
\hypersetup{
    colorlinks=true,
    urlcolor=blue,
    citecolor=black,
    linkcolor=darkblue
}
\usepackage{glossaries}
\renewcommand{\glossarysection}[2][]{}
\makeglossaries
\usepackage{makeidx}
\makeindex
\makeatletter
  \renewcommand\@seccntformat[1]{\csname the#1\endcsname.\quad}
\makeatother
\def\thesection {\Alph{section}}
\def\thesubsection {\alph{subsection}}
\def\thesubsubsection {\arabic{subsubsection}}
\usepackage{titlesec}
\titlespacing{\subsection}{2em}{0pt}{0pt}

\hyphenation{op-tical net-works semi-conduc-tor}

\begin{document}
\title{Requirements Specification \\ {\large Version 1} \\[1em] {\large
    Proposal for SIQuoIA} \\[1em] Group 6}

\author{ Ryan Alcoran \and Joe Lee \and Shivalik Narad \and Nam Phan
  \and Swapna Vemparala \and Amber Wong }

\maketitle

% Glossary items here.
\newglossaryentry{term}
{
  name=term,
  description={this is the definition of the word {\it term}}
}


\section{Preface}
This is a proposal for the Simple Intelligence Quotient Increasing
Application (SIQuoIA), a platform for running and managing quiz games.

\section{Introduction}
{\bf Purpose:} SIQuoIA provides users with a web-based quiz
platform. It will include a quiz program which prompts users with
multiple choice questions with four possible answers. One and only one
of those answers is correct.

{\bf Scope:} All users will be allowed to submit quiz packets to be
added to the pool of quizzes available to other users.

Only users of the application who have sufficient privileges, such as
moderators or instructors, are allowed to approve and add
user-submitted quiz packets.

SIQuoIA will keep track of users' progress on quizzes so that if a
user logs off or is disconnected before a quiz is complete, that quiz
can be resumed when the user logs back in to the system. Users' scores
are also recorded.

Quizzes will not be timed.

Guest users will be allowed to log in, but their scores will not be
recorded and progress on incomplete quizzes can not be resumed once
the user leaves the system.

New users will be able to register for an account.

There will three user privilege levels: student, instructor, and
moderator. Moderators will have the authority create new student and
instructor accounts, and to assign and change the privilege level of a
user. Instructors will have the authority to create new student user
accounts.

\section{Glossary}
\printglossaries

\section{User requirements definition}
Users of SIQuoIA should have a working knowledge of how to use a web
browser, how to register for an account.

\section{System architecture}
To be submitted in v2 of the proposal.

\section{System requirements specification}

\section{System models}
To be submitted in v2 of the proposal.

\section{System evolution}
Sneaking in the word {\it \gls{term}} again.

\section{Appendices}

\subsection{Hardware requirements}

\printindex


%\begin{thebibliography}{1}

%\bibitem{IEEEhowto:kopka}
%H.~Kopka and P.~W. Daly, \emph{A Guide to \LaTeX}, 3rd~ed.\hskip 1em plus
%  0.5em minus 0.4em\relax Harlow, England: Addison-Wesley, 1999.

%\end{thebibliography}


\end{document}


