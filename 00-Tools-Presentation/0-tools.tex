\documentclass{beamer}
\title{Assignment 1: Tools}
%\subtitle{}
\author{$<$Joe Lee$>$}
\subject{Tools}
\usetheme{Darmstadt}
\usecolortheme{beaver}
\useoutertheme{infolines}
\useinnertheme{circles}
\usepackage{comment}

\setbeamertemplate{footline} { 
  \leavevmode
  \hbox{\begin{beamercolorbox}[wd=.5\paperwidth, ht=2.5ex, dp=1.125ex,
        leftskip=.3cm, rightskip=.3cm plus1fill]{author in head/foot}
      \usebeamerfont{author in head/foot}\insertshortdate
  \end{beamercolorbox}
  \begin{beamercolorbox}[wd=.5\paperwidth,ht=2.5ex, dp=1.125ex, 
        leftskip=.3cm plus1fill, rightskip=.3cm]{title in head/foot}
    \usebeamerfont{title in head/foot}\insertshortauthor
  \end{beamercolorbox}}
  \vskip0pt
}

\begin{document}
\maketitle
\frame{
  \transdissolve[duration=0.25]
  \frametitle{The Team}
  \framesubtitle{Ryan Alcoran, Joe Lee, Shivalik Narad, Nam Phan, Swapna Vemparala, Amber Wong}
  \begin{center}
    {\Huge Team Q 06}\\[1em]
  \end{center}
  \pause
  \begin{itemize}
    \item Lee, Joe \{Business Manager\}
      \pause
    \item Vemparala, Swapna \{Project Manager\}
    \pause
    \item Wong, Amber \{Risk Manager\}
    \pause
    \item Narad, Shivalik \{Development Manager\}
      \pause
    \item Phan, Nam \{Development Manager\}
      \pause
        \item Alcoran, Ryan \{Test Manager\}
  \end{itemize}
}

%Introduction
\frame{
  \transdissolve[duration=0.25]
  \frametitle{Software Configuration Management (SCM)}
  \framesubtitle{Tracking and controlling changes in software}
  \begin{center}
  	{\Huge Language: Java} \\[1em]
{\small"Java is an object-oriented programming language developed by Sun Microsystems in the early 1990s. The language is very similar in syntax to C and C++ but it has a simpler object model and fewer low-level facilities."} \\[2em]
  	\pause
    {\Huge Tools} \\[1em]
    \pause
    \begin{itemize}
      \item Software Configuration Management: {\bf EGit}
        
      \item Software Hosting Facility: {\bf Github}
        
        \item Standalone Bug Tracker: {\bf Codebeamer}
     
        \item Editor and IDE: {\bf Eclipse}
       
        \item  Project Management: {\bf OpenProject}

    \end{itemize}
   
  \end{center}
}

%Egit
\frame{
  \transdissolve[duration=0.25]
  \frametitle{Software Configuration Management (SCM)}
  \framesubtitle{Tracking and controlling changes in software}
  \begin{center}
    {\Huge EGit} \\[1em]
    \pause
    "The EGit project is implementing Eclipse tooling on top of a Java implementation of Git."\\[2em]
    \pause
    \begin{itemize}
      \item free, open source, designed for branching and merging
     
      \item decentralized model (everyone has own copy of repository, which can later be merged)
     
      \item Can be integrated into Eclipse, using a plug-in.
    \end{itemize}
  \end{center}
}

%Egit comparison
\frame{
  \transdissolve[duration=0.25]
  \frametitle{Software Configuration Management (SCM)}
  \framesubtitle{Tracking and controlling changes in software}
  \begin{center}
    {\Huge EGit compare to Others} \\[1em]
    \pause
        
    \begin{flushleft}
    {\Large Subversion}
	\end{flushleft}
	\pause
    \begin{itemize}
      \item free, open source, actively developed
       
      \item easier to use than Git (more tools available for non-technical users, error messages are easier to understand)
      
      \item centralized model (everyone has a working copy and changes are submitted to central repository)
      \pause
    \end{itemize}
    
    \begin{flushleft}
    {\Large LibreSource}
    \pause
	\end{flushleft}
	
    \begin{itemize}
      \item free, open source, maintained and new features under development
       
    \end{itemize}
  \end{center}
}

%Egit Installation
\frame{
  \transdissolve[duration=0.25]
  \frametitle{Software Configuration Management (SCM)}
  \framesubtitle{Tracking and controlling changes in software}
  \begin{center}
    {\Huge EGit Installation} \\[1em]
    \pause
    \begin{figure}[h!]   
      \includegraphics[width=0.8\textwidth]{Screenshots/egit/egit1.PNG}
  \caption{Instruction to install Egit into Eclispe}
\end{figure}
	
  \end{center}
}

\frame{
  \transdissolve[duration=0.25]
  \frametitle{Software Configuration Management (SCM)}
  \framesubtitle{Tracking and controlling changes in software}
  \begin{center}
    {\Huge EGit Installation} \\[1em]
    \begin{figure}[h!]   
      \includegraphics[width=0.8\textwidth]{Screenshots/egit/egit2.PNG}
  \caption{Instruction to install Egit into Eclispe}
\end{figure}
	
  \end{center}
}

%Github
\frame{
  \transdissolve[duration=0.25]
  \frametitle{Software hosting facilities}
  \begin{center}
    {\Huge GitHub} \\[1em]
    \pause
"GitHub is a web-based hosting service for software development projects that use the Git revision control system"
  \end{center}
  \begin{itemize}
  	\item Free for open source, paid for private.
  	\pause
  	\item Ad-free, not run on all free software.
  \end{itemize}
}

%Github comparison
\frame{
  \transdissolve[duration=0.25]
  \frametitle{Software hosting facilities}
  \begin{center}
    {\Huge Github compare to Others} \\[1em]
    \pause
        
    \begin{flushleft}
    {\Large Google Code}
	\end{flushleft}
	\pause
    \begin{itemize}
      \item Free. For open-source projects only.

      
      \item Ad-free, not run on all free software
      \pause
    \end{itemize}
    
    \begin{flushleft}
    {\Large Tigris}
	\end{flushleft}
	\pause
    \begin{itemize}
      \item Restricted to collaborative software development tools.
     
      \item Not ad-free
       
    \end{itemize}
  \end{center}
}
%Github installation
\frame{
  \transdissolve[duration=0.25]
  \frametitle{Software hosting facilities}
  \begin{center}
    {\Huge GitHub Installation} \\[1em]
    \pause
   \begin{figure}[h!]   
      \includegraphics[width=0.8\textwidth]{Screenshots/github/github1.PNG}
  \caption{Github web UI and sample Repository}
\end{figure}

   \end{center}
}
\frame{
  \transdissolve[duration=0.25]
  \frametitle{Software hosting facilities}
  \begin{center}
    {\Huge GitHub Installation} \\[1em]
   \begin{figure}[h!]   
      \includegraphics[width=0.7\textwidth]{Screenshots/github/github2.PNG}
  \caption{Inside a repository and how to clone it}
\end{figure}

   \end{center}
}
\frame{
  \transdissolve[duration=0.25]
  \frametitle{Software hosting facilities}
  \begin{center}
    {\Huge GitHub Installation} \\[1em]
   \begin{figure}[h!]   
      \includegraphics[width=0.7\textwidth]{Screenshots/github/github3.PNG}
  \caption{Github software for Windows}
\end{figure}

   \end{center}
}
\frame{
  \transdissolve[duration=0.25]
  \frametitle{Software hosting facilities}
  \begin{center}
    {\Huge GitHub Installation} \\[1em]
   \begin{figure}[h!]   
      \includegraphics[width=0.7\textwidth]{Screenshots/github/github7.PNG}
  \caption{Github software UI}
\end{figure}

   \end{center}
}
\frame{
  \transdissolve[duration=0.25]
  \frametitle{Software hosting facilities}
  \begin{center}
    {\Huge GitHub Installation} \\[1em]
   \begin{figure}[h!]   
      \includegraphics[width=0.7\textwidth]{Screenshots/github/github4.PNG}
  \caption{Manage and Commit Repo}
\end{figure}

   \end{center}
}
\frame{
  \transdissolve[duration=0.25]
  \frametitle{Software hosting facilities}
  \begin{center}
    {\Huge GitHub Installation} \\[1em]
   \begin{figure}[h!]   
      \includegraphics[width=0.7\textwidth]{Screenshots/github/github5.PNG}
  \caption{Manage and Commit Repo}
\end{figure}

   \end{center}
}
\frame{
  \transdissolve[duration=0.25]
  \frametitle{Software hosting facilities}
  \begin{center}
    {\Huge GitHub Installation} \\[1em]
   \begin{figure}[h!]   
      \includegraphics[width=0.7\textwidth]{Screenshots/github/github6.PNG}
  \caption{Result and History of Committing}
\end{figure}

   \end{center}
}
%Codebeamer
\frame{
  \transdissolve[duration=0.25]
  \frametitle{Standalone bug trackers}
  \begin{center}
    {\Huge CodeBeamer} \\[1em]
    \pause
    "CodeBeamer is a web based Collaborative Application Lifecycle Management and Requirements management tool for distributed software development" \\[2em]
   
  \end{center}
  \pause
  \begin{itemize}
    \item Proprietary, free version
   
    
    \item Backend: MySQL, Oracle, Apache Derby or Postgres
  
    \item CodeBeamer is the award winning Collaborative Requirements Management (RM) and Application Lifecycle Management (ALM) solution for distributed software development
  \end{itemize}
}
%Codebeamer compare
\frame{
  \transdissolve[duration=0.25]
  \frametitle{Standalone bug trackers}
  \begin{center}
    {\Huge CodeBeamer compare to Others} \\[1em]
    \pause
       
  \end{center}
   
    {\Large Bugzilla}
	
	\pause
    \begin{itemize}
      \item MPL (free, open source, developed/maintained by Mozilla Foundation)
       
      \item backend: MySQL, Oracle, PostgreSQL, SQLite
      \pause
      
    \end{itemize}
    
    
    {\Large Google Code Hosting}

	\pause
    \begin{itemize}
      \item Proprietary, hosted; available for open source projects (by Google Code)
       
      \item BigTable backend (also proprietary)
    \end{itemize}
}
%Codebeamer installation
\frame{
  \transdissolve[duration=0.25]
  \frametitle{Standalone bug trackers}
  \begin{center}
    {\Huge CodeBeamer Installation} \\[1em]
    \pause
       \begin{figure}[h!]   
      \includegraphics[width=0.7\textwidth]{Screenshots/codeBeamer/codeBeamer1.PNG}
  \caption{Download}
\end{figure}
       
  \end{center}    
}
\frame{
  \transdissolve[duration=0.25]
  \frametitle{Standalone bug trackers}
  \begin{center}
    {\Huge CodeBeamer Installation} \\[1em]
   
       \begin{figure}[h!]   
      \includegraphics[width=0.7\textwidth]{Screenshots/codeBeamer/codeBeamer2.PNG}
  \caption{Download}
\end{figure}
       
  \end{center}    
}
\frame{
  \transdissolve[duration=0.25]
  \frametitle{Standalone bug trackers}
  \begin{center}
    {\Huge CodeBeamer Installation} \\[1em]
    
       \begin{figure}[h!]   
      \includegraphics[width=0.7\textwidth]{Screenshots/codeBeamer/codeBeamer6.PNG}
  \caption{Download}
\end{figure}
       
  \end{center}    
}
\frame{
  \transdissolve[duration=0.25]
  \frametitle{Standalone bug trackers}
  \begin{center}
    {\Huge CodeBeamer Installation} \\[1em]
    
       \begin{figure}[h!]   
      \includegraphics[width=0.7\textwidth]{Screenshots/codeBeamer/codeBeamer3.PNG}
  \caption{Web UI in localhost}
\end{figure}
       
  \end{center}    
}
\frame{
  \transdissolve[duration=0.25]
  \frametitle{Standalone bug trackers}
  \begin{center}
    {\Huge CodeBeamer Installation} \\[1em]
    
       \begin{figure}[h!]   
      \includegraphics[width=0.7\textwidth]{Screenshots/codeBeamer/codeBeamer4.PNG}
  \caption{Register User}
\end{figure}
       
  \end{center}    
}
\frame{
  \transdissolve[duration=0.25]
  \frametitle{Standalone bug trackers}
  \begin{center}
    {\Huge CodeBeamer Installation} \\[1em]
    
       \begin{figure}[h!]   
      \includegraphics[width=0.7\textwidth]{Screenshots/codeBeamer/codeBeamer5.PNG}
  \caption{User's Management UI}
\end{figure}
       
  \end{center}    
}
%IDEs: Netbeans and Eclipse	
\frame{
  \transdissolve[duration=0.25]
  \frametitle{Editors and IDEs}
\framesubtitle{Netbeans, Eclipse, GNU Emacs}
  \pause
  \begin{center}
  \begin{itemize}
    \item {\bf NetBeans IDE}
      \pause
      \begin{itemize}
        \item Dual-licensed under Common Development and Distribution License (CDDL) v1.0 and GNU GPLv2, with some components released other third-party licenses
          \pause
        \item Create and debug rich web and mobile applications using the latest HTML5, \pause JavaScript, \pause and CSS3 standards.
          \pause
        \item Page inspector
          \pause
        \item CSS style editor
          \pause
        \item JavaScript editor
          \pause
        \item JavaScript debugger
          \pause
        \item Support for PHP, Java, C, C++
          \pause
      \end{itemize}
    \item {\bf Eclipse}
      \pause
      \begin{itemize}
        \item License: MPL
          \pause
        \item IDE framework
          \pause
        \item Tools framework\\[1em]
      \end{itemize}
      \pause
  \end{itemize}
  \end{center}
}
%IDEs: GNU Emacs
\frame{
  \transdissolve[duration=0.25]
  \frametitle{Editors and IDEs}
  \framesubtitle{Netbeans, Eclipse, GNU Emacs}
  \begin{itemize}
    \item {\bf GNU Emacs}
      \pause
      \begin{itemize}
        \item License: GNU GPLv3
          \pause
        \item Text editor
          \pause
        \item Extensible
          \pause
        \item Customizable
      \end{itemize}
  \end{itemize}
}
%Eclipse Installation
\frame{
  \transdissolve[duration=0.25]
  \frametitle{Editors and IDEs}
  \framesubtitle{http://www.eclipse.org/downloads/}
	\begin{center}
    {\Huge Eclipse Installation}\\[1em]
    \pause
     \begin{figure}[h!]   
      \includegraphics[width=0.7\textwidth]{Screenshots/eclipse/eclipse1.PNG}
  \caption{Download}
\end{figure}
    \end{center}
}
\frame{
  \transdissolve[duration=0.25]
  \frametitle{Editors and IDEs}

	\begin{center}
    {\Huge Eclipse Installation}\\[1em]

     \begin{figure}[h!]   
      \includegraphics[width=0.7\textwidth]{Screenshots/eclipse/eclipse2.PNG}
  \caption{Workspace}
\end{figure}
    \end{center}
}

\frame{
  \transdissolve[duration=0.25]
  \frametitle{Editors and IDEs}

	\begin{center}
    {\Huge Eclipse Installation}\\[1em]

     \begin{figure}[h!]   
      \includegraphics[width=0.7\textwidth]{Screenshots/eclipse/eclipse3.PNG}
  \caption{Create new project}
\end{figure}
    \end{center}
}

\frame{
  \transdissolve[duration=0.25]
  \frametitle{Editors and IDEs}

	\begin{center}
    {\Huge Eclipse Installation}\\[1em]

     \begin{figure}[h!]   
      \includegraphics[width=0.7\textwidth]{Screenshots/eclipse/eclipse4.PNG}
  \caption{Create new project}
\end{figure}
    \end{center}
}
\frame{
  \transdissolve[duration=0.25]
  \frametitle{Editors and IDEs}

	\begin{center}
    {\Huge Eclipse Installation}\\[1em]

     \begin{figure}[h!]   
      \includegraphics[width=0.7\textwidth]{Screenshots/eclipse/eclipse6.PNG}
  \caption{Running Program}
\end{figure}
    \end{center}
}
\frame{
  \transdissolve[duration=0.25]
  \frametitle{Editors and IDEs}

	\begin{center}
    {\Huge Eclipse Installation}\\[1em]

     \begin{figure}[h!]   
      \includegraphics[width=0.7\textwidth]{Screenshots/eclipse/eclipse5.PNG}
  \caption{Error Logs}
\end{figure}
    \end{center}
}
%OpenProj
\frame{
  \transdissolve[duration=0.25]
  \frametitle{Project Management Tools}
  \begin{center}
    {\Huge OpenProj}\\[1em]
    \pause
    "OpenProj is an open source project management software application. It intends to be a complete desktop replacement for Microsoft Project. It runs on the Java platform, allowing it to run on a variety of different operating systems"\\[2em]
    \pause
    \begin{itemize}
        \item License: Free, open source project management application that supports project timelines, issue tracking, wiki, document management, time and cost reporting, code management, Scrum, and more.
        \item Earned Value costing  
        \item Gantt chart
          
        \item Resource Breakdown Structure (RBS) chart
       
      \end{itemize}
  \end{center}
}
%OpenProj vs. Microsoft Project
\frame{
  \transdissolve[duration=0.25]
  \frametitle{Project Management Tools}
  \begin{center}
    {\Huge OpenProj vs. Microsoft Project}\\[1em]
    \pause
    \end{center}
    {\Large Similar}
    \begin{itemize}
    	\item user interface and approach to construction of project plan
        \item create workbench breakdown structure
        \item assign resources  
        \pause
      \end{itemize}
  {\Large Different}
    \begin{itemize}
    	\item OpenProj can link upwards with methods (inserting tasks is more difficult than with Microsoft Project)

        \item OpenProj can just create resources (have to do so in the resource sheet)
        \item OpenProj lacks a more detailed view and project reports that is available with Microsoft Project  
        
      \end{itemize}
}
%OpenProj installation
\frame{
  \transdissolve[duration=0.25]
  \frametitle{Project Management Tools}
  \begin{center}
    {\Huge OpenProj Installation}\\[1em]
    \pause
     \begin{figure}[h!]   
      \includegraphics[width=0.7\textwidth]{Screenshots/openProj/openProj1.PNG}
  \caption{Download}
\end{figure}
    \end{center}
   
}
\frame{
  \transdissolve[duration=0.25]
  \frametitle{Project Management Tools}
  \begin{center}
    {\Huge OpenProj Installation}\\[1em]
	\begin{itemize}
			\item Step 1 – Double click and open the downloaded file. A dialog box will appear.
			\item Step 2 – Hit run button on the dialog box.
			\item Step 3 – Choose the directory where you want to install. And click next.
			\item Step 4 – click install
			\item Step 5 – click Finish. And your installation will be complete.
	
	\end{itemize}
     
    \end{center}
   
}
\frame{
  \transdissolve[duration=0.25]
  \frametitle{Project Management Tools}
  \begin{center}
    {\Huge OpenProj Installation}\\[1em]

     \begin{figure}[h!]   
      \includegraphics[width=0.7\textwidth]{Screenshots/openProj/openProj2.PNG}
  \caption{OpenProj's UI}
\end{figure}
    \end{center}
   
}
\frame{
  \transdissolve[duration=0.25]
  \frametitle{Project Management Tools}
  \begin{center}
    {\Huge OpenProj Installation}\\[1em]

     \begin{figure}[h!]   
      \includegraphics[width=0.7\textwidth]{Screenshots/openProj/openProj3.PNG}
  \caption{Create new Project}
\end{figure}
    \end{center}
   
}
\frame{
  \transdissolve[duration=0.25]
  \frametitle{Project Management Tools}
  \begin{center}
    {\Huge OpenProj Installation}\\[1em]

     \begin{figure}[h!]   
      \includegraphics[width=0.7\textwidth]{Screenshots/openProj/openProj4.PNG}
  \caption{A project with empty sheet}
\end{figure}
    \end{center}
   
}
\frame{
  \transdissolve[duration=0.25]
  \frametitle{Project Management Tools}
  \begin{center}
    {\Huge OpenProj Installation}\\[1em]

     \begin{figure}[h!]   
      \includegraphics[width=0.7\textwidth]{Screenshots/openProj/openProj5.PNG}
  \caption{A project with details and timeline}
\end{figure}
    \end{center}
   
}

%References
\frame{
  \transdissolve[duration=0.25]
  \frametitle{References}
  \begin{center}
\begin{itemize}
	\item \url{http://en.wikipedia.org/wiki/Comparison_of_revision_control_software}
	\item \url{http://en.wikipedia.org/wiki/Comparison_of_open-source_software_hosting_facilities}
	\item \url{http://en.wikipedia.org/wiki/Comparison_of_issue-tracking_systems}
	\item \url{http://en.wikipedia.org/wiki/OpenProj\#Comparison_to_Microsoft_Project}
	\item \url{http://looble.org/git-vs-svn-which-is-better/}
	\item \url{http://slashdot.org/topic/bi/visual-studio-vs-eclipse-a-programmers-matchup/}
	\item \url{http://www.programmers.kz/programming/java/articles_java/14715-some-background-information-on-java-and-object-orientation.html}
\end{itemize}
    \end{center}
    
}

%References
\frame{
  \transdissolve[duration=0.25]
  \begin{center}
   {\Huge Team Q 06}\\[1em]
    {\Huge Thank you!}\\[2em]

    \end{center}
    
}
\end{document}
